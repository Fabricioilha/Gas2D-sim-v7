\documentclass[12pt,a4paper]{article}
\usepackage[utf8]{inputenc}
\usepackage[brazil]{babel}
\usepackage{amsmath, amssymb}
\usepackage{geometry}
\usepackage{graphicx}
\usepackage{physics}
\usepackage{siunitx}
\usepackage{setspace}
\usepackage{booktabs}

\geometry{margin=2.5cm}
\setstretch{1.3}

\begin{document}

\begin{center}
\LARGE \textbf{Simulação Numérica de um Gás Ideal 2D com Aquecimento pela Base}\\[0.5cm]
\large Relatório técnico — Física computacional
\end{center}

\section*{1. Descrição geral}

Simulação numérica de um gás ideal em um recipiente bidimensional, sem colisões entre partículas e com colisões perfeitamente elásticas nas paredes laterais e no topo. 
As colisões com a base acrescentam energia às partículas, funcionando como uma \textbf{fonte de energia externa}.
O objetivo é analisar a evolução térmica e dinâmica do sistema, incluindo o cálculo e o gráfico da posição e velocidade do \textbf{centro de massa} ao longo do tempo.

\section*{2. Fundamentos físicos e matemáticos}

\subsection*{2.1 Velocidade escalar}
A velocidade escalar (módulo da velocidade vetorial) é dada por:
\begin{equation}
    V = \sqrt{V_x^2 + V_y^2}
\end{equation}

\subsection*{2.2 Energia cinética individual}
A energia cinética de uma partícula com massa $m$ é:
\begin{equation}
    E_c = \frac{1}{2} m (V_x^2 + V_y^2)
\end{equation}

\subsection*{2.3 Teorema da equipartição de energia}
O teorema da equipartição estabelece que a energia média por partícula é:
\begin{equation}
    \langle E_c \rangle = \frac{f}{2} k_B T
\end{equation}
onde:
\begin{itemize}
    \item $f$ = número de graus de liberdade,
    \item $k_B$ = constante de Boltzmann,
    \item $T$ = temperatura em kelvin.
\end{itemize}

Para um sistema bidimensional ($f = 2$):
\begin{equation}
    \langle E_c \rangle = k_B T
\end{equation}

\textbf{Importante:} não se pode igualar diretamente $\langle E_c \rangle$ à energia cinética individual $E_c$ de uma partícula, pois $\langle E_c \rangle$ representa a \textbf{média} de muitas partículas. 
Para relacionar ambas, deve-se tomar o somatório das energias individuais e dividir por $N$, o número total de partículas.

\section*{3. Temperatura efetiva do sistema}

A energia cinética total é:
\begin{equation}
    K_{\text{total}} = \sum_{i=1}^{N} \frac{1}{2} m (V_{xi}^2 + V_{yi}^2)
\end{equation}

Pela equipartição:
\begin{equation}
    K_{\text{total}} = \frac{f}{2} N k_B T
\end{equation}

Isolando a temperatura:
\begin{equation}
    \boxed{T_{\text{efetiva}} = \frac{2}{f N k_B} \sum_{i=1}^{N} \frac{1}{2} m (V_{xi}^2 + V_{yi}^2)}
\end{equation}

\section*{4. Velocidade RMS (Root Mean Square)}

Como as velocidades podem ser positivas ou negativas, a média simples $\langle V \rangle$ pode ser zero, mesmo que todas as partículas estejam se movendo rapidamente. 
A velocidade RMS mede a intensidade real do movimento:
\begin{equation}
    V_{\text{rms}} = \sqrt{\frac{1}{N} \sum_{i=1}^{N} (V_{xi}^2 + V_{yi}^2)}
\end{equation}

\noindent
Relação entre $V_{\text{rms}}$ e a temperatura (via equipartição):
\begin{equation}
    \frac{1}{2} m \langle V_x^2 + V_y^2 \rangle = k_B T
\end{equation}

Logo:
\begin{equation}
    \boxed{V_{\text{rms}} = \sqrt{\frac{2 k_B T}{m}}}
\end{equation}

Quanto maior a temperatura, maior o valor médio de $V_{\text{rms}}$, indicando que as partículas estão mais agitadas.

\section*{5. Injeção de energia na base}

A cada colisão com a base, uma partícula recebe energia adicional:
\begin{equation}
    Q_{\text{per hit}} = c \, k_B \, T_{\text{base}}
\end{equation}
onde:
\begin{itemize}
    \item $c$ = parâmetro de eficiência de transferência de energia,
    \item $k_B$ = constante de Boltzmann,
    \item $T_{\text{base}}$ = temperatura da base em kelvin.
\end{itemize}

O \textbf{fator de energia} é calculado como:
\begin{equation}
    F = \sqrt{\frac{E_{\text{nova}}}{E_{\text{atual}}}}
\end{equation}
Esse fator ajusta o módulo total da velocidade, aumentando proporcionalmente $V_x$ e $V_y$ de forma isotrópica (em ambas as direções).

\section*{6. Movimento das partículas}

O movimento de cada partícula é modelado por um \textbf{Movimento Retilíneo Uniforme (MRU)}, com atualização temporal usando o \textbf{método de Euler}:
\begin{align}
    x(t + \Delta t) &= x(t) + V_x(t)\,\Delta t \\
    y(t + \Delta t) &= y(t) + V_y(t)\,\Delta t
\end{align}

O método de Euler é um integrador de primeira ordem, simples e eficiente para simulações com pequenos intervalos de tempo ($\Delta t$ pequeno).

\section*{7. Centro de massa do sistema}

O centro de massa representa a posição média ponderada pela massa de todas as partículas. Para $N$ partículas de massas $m_i$ e posições $(x_i, y_i)$:
\begin{align}
    x_{CM} &= \frac{\sum_{i=1}^{N} m_i x_i}{\sum_{i=1}^{N} m_i}, &
    y_{CM} &= \frac{\sum_{i=1}^{N} m_i y_i}{\sum_{i=1}^{N} m_i}
\end{align}

Se todas as partículas têm a mesma massa $m$, as massas se cancelam:
\begin{align}
    x_{CM} &= \frac{1}{N} \sum_{i=1}^{N} x_i, &
    y_{CM} &= \frac{1}{N} \sum_{i=1}^{N} y_i
\end{align}

A velocidade do centro de massa é obtida como:
\begin{align}
    v_{CM,x} &= \frac{1}{N} \sum_{i=1}^{N} V_{xi}, &
    v_{CM,y} &= \frac{1}{N} \sum_{i=1}^{N} V_{yi}
\end{align}

\section*{8. Interpretação física}

\begin{itemize}
    \item \textbf{Temperatura efetiva:} representa a agitação média das partículas (energia interna do sistema). É uma grandeza escalar.
    \item \textbf{Centro de massa:} representa a posição média do sistema. É uma grandeza vetorial.
    \item \textbf{Velocidade do centro de massa:} indica o movimento global do conjunto de partículas. Também é vetorial.
\end{itemize}

\section*{9. Conclusão}

O sistema descrito realiza uma \textbf{simulação numérica de um gás ideal bidimensional} sob aquecimento pela base, utilizando o \textbf{método de Euler} para integração temporal.  
A temperatura efetiva do sistema é obtida pela média da energia cinética das partículas, enquanto o centro de massa e sua velocidade descrevem o comportamento global do conjunto.  
Com o aumento da energia injetada pela base, espera-se observar um crescimento de $T_{\text{efetiva}}$ e uma variação ascendente na coordenada $y_{CM}$, análoga ao efeito de \textbf{convecção térmica}.

\end{document}
